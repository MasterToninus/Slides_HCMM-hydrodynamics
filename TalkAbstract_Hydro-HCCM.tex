%+------------------------------------------------------------------------+
%| Abstract: short presentation on my paper 1805.01696
%| Event: Young researchers GMC workshop - Coimbra, Dec. 2018
%| Author: Antonio miti
%+------------------------------------------------------------------------+


\documentclass[11pt,a4paper,twoside]{article}

\usepackage{fancyhdr}
%\usepackage[T1]{fontenc}
\usepackage{amsfonts}

%%%%%%%%%%%%%%%%%%%%%%%%%%%%%%%%%%%%%%%%%%%%%%%%%
%             SIZE             %
%%%%%%%%%%%%%%%%%%%%%%%%%%%%%%%%%%%%%%%%%%%%%%%%%


\parskip=0.5ex
\oddsidemargin= 0.35cm
\evensidemargin= 0.35cm

\parindent=1.5em
\textheight=23.0cm
\textwidth=15.5cm


%%%%%%%%%%%%%%%%%%%%%%%%%%%%%%%%%%%%%%%%%%%%%%%%%
%%%%%%%%%%%%%%%%%%%%%%%%%%%%%%%%%%%%%%%%%%%%%%%%%

%%%%%%%%%%%%%%%%%%%%%%%%%%%%%%%%%%%%%%%%%%%%%%%%%
%             ABSTRACT             %
%%%%%%%%%%%%%%%%%%%%%%%%%%%%%%%%%%%%%%%%%%%%%%%%%


\def\sj{\vskip.6cm}

\def \yourtitle #1{{\Large \bf #1} \par \vspace{10pt}}

\def \speaker #1#2#3#4{
\textbf{#1 #2} \par #3 \par \texttt{#4}
\par
\vspace{7pt}
}

\def \author #1#2#3#4{
#1 #2 \par #3 \par \texttt{#4}
\par
\vspace{7pt}
}


\def \thanks #1{\footnote{#1}}

\def \abstracttext{
\vspace{-2pt}
\noindent
%\hrulefill
\vspace{8pt}
}


%%%%%%%%%%%%%%%%%%%%%%%%%%%%%%%%%%%%%%%%%%%%%%%%%
%%%%%%%%%%%%%%%%%%%%%%%%%%%%%%%%%%%%%%%%%%%%%%%%%

\begin{document}
\pagestyle{fancy} \fancyhead[RO]{{\large 13th Young Researchers Workshop } \\  on Geometry, Mechanics and Control}
\fancyhead[LE]{{\large 13th YRW} \\ 13th YRWorkshop  on Geometry, Mechanics and Control}
\fancyhead[LO]{} \fancyhead[RE]{}

\yourtitle{A Homotopy co-momentum Map in Hydrodynamics} % Insert here the title of your work

% Insert below the speaker's first name, last name, affiliation, and Email address
\speaker{Antonio Michele}
{Miti}
{Universit\' a Cattolica del Sacro Cuore, Department of Mathematics and Physics, Brescia
	\phantom{AA} KU Leuven, Department of Mathematics, Leuven}
{antoniomichele.miti@kuleuven.be}
%
% Insert below the first name, last name, affiliation, and Email address for any coauthor of your work
\author{Mauro}
{Spera}
{Universit\' a Cattolica del Sacro Cuore, Department of Mathematics and Physics, Brescia}
{mauro.spera@unicatt.it}
%
%\author{co-author 2 first name}
%{co-author 2 last name\thanks{co-author 2 wishes to thank someone}}
%{co-author 2 affiliation}
%{co-author 2 Email}

\abstracttext

In this talk, based on joint work with M. Spera \cite{firsticle}, we investigate some connections between multisymplectic geometry and hydrodynamics.
 \\
 After a brief review of the basic definitions of \emph{multisymplectic manifold} and \emph{Lie-$\infty$ algebra of observables} \cite{R}, we recall the notion of an \emph{Homotopy co-momentum map} \cite{CFRZ} and realize an explicit construction of such object in the case of $\mathbb{R}^3$ which is relevant to hydrodynamics.
 In this way, we are able to reinterpret the so-called \emph{Rasetti-Regge currents} \cite{RR}, introduced in the contest of vortex dynamics, as momenta associated to the vorticity.
 \\
 Time permitting, we shall discuss a generalization of the above construction in the case of \emph{perfect fluid} on compact, oriented Riemannian manifold satisfying appropriate cohomological conditions. %intermediate cohomology groups vanish $H^k(M)= 0\,,\, 1<k<n-2$
 \\
 The former construction finds an application in knot theory starting from the observation that $n$-links can be related to suitable \emph{conserved quantities} \cite{RWZ}.

\begin{thebibliography}{99}
	\bibitem{firsticle}
		Antonio Michele Miti and Mauro Spera.
		\newblock \emph{On some (multi)symplectic aspects of link invariants}, 2018;
		\newblock arXiv:1805.01696.
	%
	\bibitem{R}
		Christopher L. Rogers.
		\newblock \emph{L-infinity algebras from multisymplectic geometry}, 2010,
		\newblock Lett. Math. Phys. 100 (2012), 29-50;
		\newblock arXiv:1005.2230;
		\newblock DOI:10.1007/s11005-011-0493-x.
	%
	\bibitem{CFRZ}
		Martin Callies, Yael Fregier, Christopher L. Rogers and Marco Zambon.
		\newblock \emph{Homotopy moment maps}, 2013;
		\newblock Adv. in Math. 303 (2016), 954-1043;
		\newblock arXiv:1304.2051;
		\newblock DOI:10.1016/j.aim.2016.08.012.
	%
	\bibitem{RR}
		Mario Rasetti, Tullio Regge.
		\newblock \emph{Vortices in He II, current algebras and quantum knots}, 1975,
		\newblock PHYSICA A. 80. (1975), 217-233;
		\newblock DOI:10.1016/0378-4371(75)90105-3. 
	%
	\bibitem{RWZ}
		Leonid Ryvkin, Tilmann Wurzbacher and Marco Zambon.
		\newblock \emph{Conserved quantities on multisymplectic manifolds}, 2016;
		\newblock arXiv:1610.05592.
\end{thebibliography}


\end{document}
